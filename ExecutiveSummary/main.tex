
%-----------------------------------------------------------------------------------
%-----------------------------------------------------------------------------------
%                                      PREAMBLE
%-----------------------------------------------------------------------------------
%-----------------------------------------------------------------------------------
\documentclass{article}
%-----------------------------------------------------------------------------------
% GENERIC PACKAGES
\usepackage[utf8]{inputenc}
\usepackage{graphicx}
\usepackage{hyperref}
\usepackage{setspace}
\usepackage[T1]{fontenc}
\usepackage{multicol}
\usepackage{blindtext}
%-----------------------------------------------------------------------------------
% GEOMETRY
\usepackage{geometry}
\geometry{
    left=25mm,
    right=25mm,
    top=25mm,
    bottom=25mm
    }
\onehalfspacing
%-----------------------------------------------------------------------------------
% ACRONYMS
\usepackage{acro}
\DeclareAcronym{hri}{
  short=HRI,
  long=Human-Robot Interaction,
}
\DeclareAcronym{mis}{
  short=MIS,
  long=minimally Invasive Surgery,
}
\DeclareAcronym{ramis}{
  short=RAMIS,
  long=Robot-Assisted Minimally Invasive Surgery,
}
%-----------------------------------------------------------------------------------
% NEWCOMMANDS AND RENEWCOMMANDS

%-----------------------------------------------------------------------------------

% TITLE
\title{
\begin{figure}[h!]
\centering
\includegraphics[width=0.5\textwidth]{images/logo_polimi_scritta2.eps}
\end{figure}
\textbf{Implementation and assessment of Assistance-as-Needed Virtual Fixtures for Surgical Training with a \textit{daVinci} surgical robot: an experimental study}
\\
\vspace{0.5cm}\large{\textit{Politecnico di Milano - Master of Science in Biomedical Engineering}}
\\
\vspace{0.5cm}\textit{\small{Academic year 2022/2023}}\vspace{0.5cm}\\
{\large \textbf{Candidate: \textit{Alberto Rota}}\\
\textbf{Supervisor: \textit{prof. Elena de Momi}}}}
\author{}
\date{}

%-----------------------------------------------------------------------------------
%-----------------------------------------------------------------------------------
%                                      BODY
%-----------------------------------------------------------------------------------
%-----------------------------------------------------------------------------------
\begin{document}
% TITLE
\maketitle
%-----------------------------------------------------------------------------------
% AUTHOR

% \vspace{1cm}
\begin{multicols}{2}

%-----------------------------------------------------------------------------------
\section{Introduction}
Since the introduction of \ac{ramis} in the healthcare market, patients have benefited from the diminishing of postoperative complications and an increase in the safety of procedures, while surgeons have been empowered by sophisticated robotic systems allowing the execution of complex tasks in adverse contexts. 

Surgical teams have been joined by robotic systems because of the number of benefits that result from the collaborative effort of the surgeon and the robot, combining the know-how and adaptation skills of the former and the high accuracy and stability of the latter. Such a synergistic perspective highlights the role of the \ac{hri} paradigm, which involves all aspects of understanding, designing, and evaluating robotic systems for use by or with humans \cite{Goodrich2007}. Most of the surgical robotics solutions on the market consist of a teleoperation console that interfaces with the practicioner and, separate from it, the surgical robot itself, which mimics the movements of the surgeon in real-time. This setup allows for higher motion accuracy, tremor filtration and magnified viewing of the surgical area; nonetheless tactile forces, friction and texture perception arez
\section{State of the art}
\section{Materials and methods}
\section{Results}
\section{Discussion}
\section{Conclusions}
%-----------------------------------------------------------------------------------
% BIBLIOGRAPHY
\bibliographystyle{plain}
\bibliography{refs.bib}
%-----------------------------------------------------------------------------------
\end{multicols}
\end{document}