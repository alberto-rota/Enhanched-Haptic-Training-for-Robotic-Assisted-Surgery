
%-----------------------------------------------------------------------------------
%-----------------------------------------------------------------------------------
%                                      PREAMBLE
%-----------------------------------------------------------------------------------
%-----------------------------------------------------------------------------------
\documentclass{article}
%-----------------------------------------------------------------------------------
% GENERIC PACKAGES
\usepackage[utf8]{inputenc}
\usepackage{graphicx}
\usepackage{hyperref}
\usepackage{setspace}
%-----------------------------------------------------------------------------------
% GEOMETRY
\usepackage{geometry}
\geometry{
    left=25mm,
    right=25mm,
    top=20mm,
    bottom=25mm
    }
\onehalfspacing
%-----------------------------------------------------------------------------------
% ACRONYMS
\usepackage{acro}
\DeclareAcronym{hri}{
  short=HRI,
  long=Human-Robot Interaction,
}
\DeclareAcronym{mis}{
  short=MIS,
  long=minimally Invasive Surgery,
}
\DeclareAcronym{ramis}{
  short=RAMIS,
  long=Robot-Assisted Minimally Invasive Surgery,
}
%-----------------------------------------------------------------------------------
% NEWCOMMANDS AND RENEWCOMMANDS

%-----------------------------------------------------------------------------------

% TITLE
\title{\textbf{Implementation and assessment of Assistance-as-Needed Virtual Fixtures for Surgical Training with a \textit{daVinci} surgical robot: an experimental study}
\\
\vspace{0.5cm}\large{\textit{Alberto Rota} : NearLab @ Politecnico di Milano}
\\
\vspace{0.4cm}\small{Project Supervisor: \textit{prof. Elena De Momi}}
}
\author{}
\date{}

%-----------------------------------------------------------------------------------
%-----------------------------------------------------------------------------------
%                                      BODY
%-----------------------------------------------------------------------------------
%-----------------------------------------------------------------------------------
\begin{document}
% TITLE
\maketitle

%-----------------------------------------------------------------------------------
\section{Scope of the study}
To implement a simulator for surgical training that employs haptic Virtual Fixtures as correction media, and to assess the effectiveness and utility of haptic Virtual Fixtures in the process of learning surgical skills, their retention and their transferability.

\paragraph*{Haptic Virtual Fixtures:} When teleoperating the surgical robot, the motors in the manipulators are activated so that a tactile/haptic force is felt by the surgeon: this force is applied with the aim of stirring the surgeon motion away from obstacles or towards objectives
\paragraph*{Skill Retention:} After surgical training, the trainee should be able to perform the surgical tasks correctly even after long periods of time away from the simulator
\paragraph*{Skill Transfer:} After surgical training conducted on a limited number of tasks, the trainee should be able to perform correctly on new never-seen-before chores.

\section{Progress}
\begin{itemize}
  \item  We successfully built the surgical training simulator. Teleoperation occurs from the manipulators of a \textit{daVinci} surgical robot; the virtual surgical scene is visible in 3D from the High-Resolution Stereo Viewers located at the teleoperation console
  \item We have constructed a tentative experimental protocol, where the performance of volunteer subjects that will receive haptic assistance will be compared with a control group where the assistance will not be provided.
  \item We have gathered promising preliminary data from a very limited population of subjects with no surgical experience
\end{itemize}

\section{Expectations}
In a context where the subjects of the experimental study (divided in an \textit{assisted} group and a \textit{control} (unassisted) group) perform multiple times the surgical tasks over the course of a few days, and where the \textbf{performance} in the execution of a task is calculated as a numerical value, it is expected that:
\begin{itemize}
  \item Subjects in the assisted group will reach their best performances earlier than subjects in the control group
  \item The best performance of subjects in the assisted group will be higher than the best performance of subjects in the control group
  \item After a "break" of a few days, the subjects in the assisted group will be able to perform the surgical tasks with a performance comparable to the one obtained in the days before the break, \textit{i.e.} they will have \textbf{successfully retained their skill}
  \item After a "break" of a few days, the subjects in the assisted group will be able to tackle new never-seen-before surgical tasks with a performance comparable to the one obtained in the days before the break, \textit{i.e.} they will \textbf{successfully have transferred their skill}
\end{itemize} 
The figure below shows graphically the expectations of the study

\begin{figure}[!h]
  \centering
  \includegraphics[width=0.95\textwidth]{expected.jpg}
  % \label{fig:expected}
\end{figure} 
\section{Further requirements}
Before proceeding with the experimental study, we must \textbf{assert the clinical relevance and compliance} of the surgical simulator that we have implemented and of the experimental protocol that we have planned. For that purpose opinions, corrections, criticism and involvement from the clinical community are encouraged and welcomed. 

\noindent We ask you to kindly dedicate approximately 30 minutes of your time to test the simulator (8 surgical tasks in total are implemented, each taking at the very very maximum 2 minutes) and to give us your feedback regarding the haptic Virtual Fixtures, the surgical simulator and the experimental protocol.
\newline \newline
{\footnotesize
Contact Alberto at: \texttt{\href{mailto:alberto2.rota@mail.polimi.it}{alberto2.rota@mail.polimi.it}} or \texttt{+39 346 214 2633}
\newline Project WebPage:\newline\noindent \href{https://github.com/alberto-rota/Virtual-Fixtures-in-Robotic-Assisted-Surgery/blob/main/README.md}{\texttt{https://github.com/alberto-rota/Virtual-Fixtures-in-Robotic-Assisted-Surgery/blob/main/README.md}
}}


%-----------------------------------------------------------------------------------


\end{document}