\documentclass[../main.tex]{subfiles}

\graphicspath{{\subfix{../images/}}}
\begin{document}

\addcontentsline{toc}{chapter}{Abstract}
\chapter*{Abstract}
The increase of surgical robotics procedures in the last decade demands a high number of trained surgeons, capable of teleoperating such advanced and complex systems and at the same time able to take advantage of the benefits of Robot-Assisted Minimally Invasive Surgery safely and effectively. Training curricula nowadays rely on Virtual Reality and simulated environments to achieve a scalable, cost-effective and comprehensive establishment of the surgical robotic skillset.

This work presents the development and validation of a haptic-enhanced Virtual Reality training simulator for the \davinci surgical robot, which is currently the most widely used surgical robotic system in the world. The simulator features and emulates 8 surgical tasks, which the trainee can realistically interact with thanks to the embedded physics engine. This virtual simulated environment features high-level haptic interfaces for robotic assistance (also known as \textit{Virtual Fixtures}) that, through the manipulators mounted at the surgical console, generate forces and torques that aim at re-directing the motion of the trainee's hands and wrists toward targets or away from obstacles. 

A validating experimental study demonstrated that the introduction of high-level haptic assistance algorithms into the training phase undergone by aspiring robotic surgeons improves performance during the training process and, crucially, promotes the transfer of the acquired skills to an unassisted surgical scenario, like the clinical one. 

Enhancing surgical training with haptic assistance algorithms is a promising approach to improve the establishment of an enriched surgical robotic skillset: this represents a valuable step toward the widespread adoption of surgical robotics in clinical practice.

\textbf{Keywords:} Surgical Robotics, Enhanced Training, Haptic Assistance, Skill Transfer

\newpage\newpage
\addcontentsline{toc}{chapter}{Sommario}
\chapter*{Sommario}
La crescita nel numero di interventi di robotica chirurgica nell'ultimo decennio richiede un numero sempre più elevato di chirurghi formati al loro utilizzo, in grado di teleoperare sistemi avanzati e complessi e allo stesso tempo di sfruttare i vantaggi della chirurgia robotica mini-invasiva in modo sicuro ed efficace. I programmi di formazione, ad oggi, si affidano alla Realtà Virtuale e alla simulazione per ottenere un training scalabile, economico e completo su tutte le competenze di robotica chirurgica.

In questo lavoro è presentato lo sviluppo e la validazione di un simulatore formativo in Realtà Virtuale ottimizzato apticamente per il robot chirurgico \davinci, attualmente il sistema più utilizzato al mondo nel campo della chirurgia mini-invasiva. Il simulatore presenta ed emula 8 operazioni chirurgiche, con le quali il praticante può interagire realisticamente grazie al motore fisico integrato. Questo ambiente virtuale simulato è dotato di interfacce aptiche di alto livello per l'assistenza robotica (note anche come \textit{Vincoli Virtuali}) che, attraverso i manipolatori montati sulla console chirurgica, generano forze e coppie che mirano a reindirizzare il movimento delle mani e dei polsi del praticante verso target predefiniti o lontano da ostacoli. 

Uno studio sperimentale di validazione ha dimostrato che l'introduzione di algoritmi di assistenza aptica di alto livello nella fase di addestramento degli aspiranti chirurghi robotici migliora le prestazioni durante il processo di addestramento e, soprattutto, favorisce il trasferimento delle competenze acquisite a uno scenario chirurgico non assistito, come quello clinico. 

Il miglioramento del training chirurgico con algoritmi di assistenza aptica è un approccio promettente per migliorare la creazione di un set di competenze chirurgiche robotiche arricchite: ciò rappresenta un passaggio cruciale verso una più diffusa adozione della robotica chirurgica nella pratica clinica.

\textbf{Parole Chiave:} Robotica Chirurgica, Formazione Avanzata, Assistenza Aptica, Trasferimento di Abilità

\end{document}