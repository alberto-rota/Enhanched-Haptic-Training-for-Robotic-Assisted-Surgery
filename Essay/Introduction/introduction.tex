\documentclass[../main.tex]{subfiles}
\graphicspath{{\subfix{../images/}}}

\begin{document}

\section{Introduction}
Since its advent in the middle of the 1980s, robotic surgery has revolutionized the healthcare industry introducing safer and more efficient solutions to the challenges of surgical practices. By embracing robotic devices in a collaborative effort, surgeons and medical practitioners have been empowered with tools engineered to enhance their skills and optimize patient outcomes. 
From dental to orthopedic to neurosurgery, almost every area of the medical panorama has now been reached by robotics, with regulatory organizations approving novel procedures with increasing frequency.

Formally, \textit{robot-assisted surgeries} are operations where the medical team works with a robotic device that actively interacts with the patient's body, manipulating tissues with precision instruments that are mounted on the machine itself. 
However, there is a major distinction between two distinct methodologies of deploying robotic assistance \cite{Hoeckelmann2015}:
\begin{itemize}
    \item \textbf{Teleoperated systems} involve a user who directly and continuously interacts with the robot, controlling its motion mainly by hand-held manipulators which replicate the surgeon's movement to the robotic actuaction
    \item \textbf{Image-guided systems} rely on the acquisition of medical imaging data (in the form of CT scans, MRIs, \textit{etc.}), later used for planning the motion of the robotic tools, without a direct and real-time interface with the surgeon
\end{itemize}
Mixed approaches are also possible, where teleoperation is accompanied by fully automated sub-tasks or where image guidance is updated in real-time by information gathered from sensors. 
In all cases, controlling the position, orientation and motion of the robotic system is, however, a responsibility of the surgeon himself, that takes advantage of the high precision potentiality of the mechanical system for achieving a less invasive, less error-prone and safer procedure. These systems are not meant to replace the physician, but rather to augment its capabilities \cite{Cleary2001}. Such a synergistic perspective highlights the role of the \ac{hri} paradigm, which involves all aspects of understanding, designing, and evaluating robotic systems for use by or with humans \cite{Goodrich2007}: in the context of teleoperated surgical robots, \ac{hri} includes all the hardware and software features that enhance the surgeon's experience, improve his performance, grant a higher level of safety for the patient and achieve better surgical outcomes.

Modern surgical robotic systems like the \textit{daVinci}\cright (Intuitive Surgical, Inc.) are extremely complex devices and, as such, demand long and extensive training programs to medical facilities and surgeons who want to operate them. In the past, surgical training was conducted on semi-realistic plastic phantoms, animals or cadavers, which other than not being a reusable resource in a lot of cases came out to be non-cost-effective solutions. More modern approaches are, instead, virtual environments where a simulated surgical scenario is re-constructed with a discrete level of realism, and where physics engines emulate the interaction between the virtual objects. A \ac{vr} environment has multiple advantages compared to the dated approaches above: infinite customizability and repeatability, non-destructiveness, easy setup, reduced costs, accurate progress monitoring, \textit{etc.} Virtual environments, also, allow and easier developing and deploying of assistive algorithms that, by running ``in parallel'' to the surgeon's teleoperation, contribute to the \ac{hri} paradigm in terms of performance and safety. 

Surgical assistance has become an impactful element in the most recent surgical robotic solution on the market, for the most part concerning visual cues super-imposed to the camera feed. For example, such visual cues may consist in the detection and localization, on the screen, of delicate surgical structures that should remain untouched by the instrumentation. Deep Learning and other modern AI-based computer vision techniques are the most useful in this context, and a few commercialized surgical robots already employ such assistive strategies.

Still, a

\subsection{Context}

\subsection{Motivation}
\textit{}

% BIBLIOGRAPHY
\bibliographystyle{unsrt}
\bibliography{refs.bib}

\end{document}